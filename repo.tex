% ctrl-alt-b でビルド
% ここからヘッダ部
\documentclass[a4paper,10pt,titlepage]{jsarticle}
%使用パッケージ設定
\usepackage[margin=20mm,includefoot]{geometry}
\usepackage[dvipdfmx]{graphicx}
\usepackage{url}
\usepackage{moreverb}
\usepackage{framed}
\usepackage{amsmath}
\usepackage{bm}
\usepackage{here}
\usepackage{amssymb}
\usepackage{listings}
\usepackage{ascmac}
\usepackage{listings,jlisting}
\usepackage{ulem}
%\usepackage[dvipdfmx]{hyperref}
%\usepackage{pxjahyper}
% 自作命令設定
\newcommand{\ttt}[1]{\texttt{#1}}
%ここからソースコードの表示に関する設定
\lstset{
  basicstyle={\ttfamily},
  identifierstyle={\small},
  commentstyle={\smallitshape},
  keywordstyle={\small\bfseries},
  ndkeywordstyle={\small},
  stringstyle={\small\ttfamily},
  frame={tb},
  breaklines=true,
  columns=[l]{fullflexible},
  numbers=left,
  xrightmargin=0zw,
  xleftmargin=3zw,
  numberstyle={\scriptsize},
  stepnumber=1,
  numbersep=1zw,
  lineskip=-0.5ex
}
%ここまでソースコードの表示に関する設定
% maketitle用の設定
\title{3D Photography using Context-aware Layered Depth Inpainting -翻訳
}
\date{掲載日:2020-04-09}
\author{作成者:Meng-Li Shih, Shih-Yang Su, Johannes Kopf, Jia-Bin Huang}

%頁番号の有無
%\pagestyle{empty}
% ここから本文
\begin{document}
% タイトルページの作成
\maketitle

\section{図の説明}
図1. ingleRGB-Dimageからの3D写真。従来の方法では、穴が開いているか(a)、あるいは離接点で伸びているか(b)のどちらかである。拡散を利用した色や深さの塗り潰しは良いのですが、あまりにも滑らかな外観になってしまいます(c)。我々のアプローチは、新しい色や奥行きのテクスチャや構造を合成することができ、よりフォトリアリスティックで斬新なビューを実現することができます(d)。

\section{Abstract}
我々は、RGB-Dの単一の入力画像を3D写真に変換する方法を提案します。我々は、基礎となる表現として明示的なピクセル接続性を持つレイヤードデプス画像を使用し、空間的文脈を認識した方法で、閉塞した領域に新しい局所的な色と奥行きのコンテンツを合成する学習ベースのインペイントモデルを提示する。得られた3D写真は、標準的なグラフィックスエンジンを用いて、モーションパララックスを用いて効率的にレンダリングすることができる。我々は、我々の手法の有効性を、より広範囲の困難な日常のシーンで検証し、最先端の技術と比較して、アーチファクトが少ないことを示しています。

\section{Introduction}
3D写真は、カメラで世界の景色を撮影し、イメージベースのレンダリング技術を使って新しい景色を合成することで、視覚的な知覚を記録し、再生産するための最も魅力的な方法です。従来の2D写真よりも劇的に没入感のある体験を提供します。バーチャルリアリティではほとんど実物に近い、視差をつけて表示すると通常のフラットディスプレイでもある程度は再現されます。
古典的なイメージベースの再構成とレンダリング技術 しかし、3D写真の撮影には、大きなベースライン[17, 59, 26, 45, 19, 12]を持つ多くの画像を含む精巧なキャプチャセットアップ、および/または特別なハードウェア(例:Lytro Immerge、Facebook Manifoldカメラ1)を必要とします。\\

最近では、携帯電話のカメラを使用し、ベースライン要件を下げることで、3D撮影のためのキャプチャをより簡単にしようとする動きが見られます[17, 18]。
 最も極端な例では、Facebook 3D Photosのような斬新な技術では、デュアルレンズのカメラ付き携帯電話で1枚のスナップショットをキャプチャするだけで、基本的にRGB-D(色と奥行き)入力画像を提供しています。\\
 この作品では、このような RGB-D 入力から新しいビューをレンダリングすることに興味を持っています。レンダリングされた新しいビューで最も顕著な特徴は、視差による閉塞感です。最近の手法は、より良い外挿を提供しようとしています。\\

ステレオマグニフィケーション [72] と最近のバリアント [52, 39] は,小型ベースラインのデュアルカメラステレオ入力から合成された前頭-並列マルチプレーン表現 (MPI) を使用しています.しかし,MPIは傾斜した表面ではアーチファクトが発生します.さらに、マルチプレーン表現の過剰な冗長性は、メモリやストレージを不完全なものにし、レンダリングにコストがかかります。\\

Facebook 3D Photosでは、レイヤードデプスイメージ(LDI)表現[48]を使用していますが、これは分散性があるためよりコンパクトで、レンダリングのために軽量なメッシュ表現に変換することができます。オクルージョン領域の色と奥行きは,モバイルデバイス上での高速ランタイムのために最適化されたヒューリスティックを使用して合成されます.特に、色の塗り潰しに等方性拡散アルゴリズムを使用していますが、これは過度に滑らかな結果を生み出し、テクスチャや構造を外挿することができません(図1c)。\\

最近のいくつかの学習ベースの方法もまた、同様の多層画像表現を使用している[7, 56]。しかし、これらの方法は、画像内のすべてのピクセルが同じ(fixされたあらかじめ決められた)数のレイヤーを持っているという意味で、「剛直な」レイヤー構造を使用します。各ピクセルでは,最初の層に最も近い表面を,次の層に2番目に近い表面を格納します.これは問題である.なぜなら,深さの不連続性を越えて層内の内容が急激に変化するため,畳み込みカーネルの受容的な識別子の局所性が破壊されるからである.\\

本研究では,RGB-D入力から3D写真を生成する新しい学習ベースの手法を提示する.
 奥行きは、デュアルカメラの携帯電話のステレオ画像から得られるか、単一のRGB画像から推定することができる [30, 28, 13]。我々はLDI表現(Facebook 3D Photosに似ている)を使用しているが,これはコンパクトであり,任意の深度複雑度の状況を扱うことができるからである.上述の「リジッド」なレイヤー構造とは異なり、我々の表現ではピクセル間の接続性を明示的に格納しています。しかし,トポロジーが標準的なテンソルよりも複雑であるため,結果として大域的なCNNを適用することは困難です.その代わりに,問題を多くの局所的な塗り潰しのサブ問題に分割し,それらを反復的に解く.各問題は局所的に画像のようなものであるため、標準的なCNNを適用することができる。合成後、インペイントされた領域をLDIに戻し、すべての深度エッジが処理されるまで再帰的アルゴリズムを実行する。\\

我々のアルゴリズムの結果は、合成されたテクスチャと閉塞した領域の構造を持つ3D写真である(図1d)。これまでのほとんどのアプローチとは異なり、我々はレイヤーのfixed数を事前に決定する必要はありません。その代わりに、我々のアルゴリズムは、入力の局所的な深さの複雑さに設計によって適応し、画像全体で変化するレイヤーの数を生成します。我々は、様々な状況で撮影された様々な写真で我々のアプローチを検証しました。\\

\section{Method}
\textbf{階層化された奥行き画像}:
我々の手法では、RGBD画像(色と深度を揃えた画像ペア)を入力として受け取り、入力ではみ出した部分に色と深度を塗り分けたレイヤードデプス画像(LDI, [48])を生成する。\\

LDI は通常の 4 つの連結画像と似ていますが、ピクセル格子の各位置に 0 から多数までの任意の数のピクセルを保持することができます。
各 LDI ピクセルには色と深度値が格納されています。オリジナルの LDI [48] とは異なり、 ピクセルの局所的な接続性を明示的に表現しています。各ピクセルには、左、右、上、下の 4 つの方向(左、右、上、下)のそれぞれに、ゼロか最大で 1 つの直接の隣人へのポインタが格納されています。\\

 LDI ピクセルは滑らかな領域内では通常の画像ピクセルのように 4 つの連結性を持っていますが,奥行きの不連続性を越えたピクセルは持っていません.\\

 LDIは
 \begin{itemize}
   \item 任意のレイヤー数を自然に扱うことができ、必要に応じて奥行きが複雑な状況にも適応できること
   \item メモリやストレージが少なく、高速にレンダリングできる軽量なテクスチャメッシュ表現に変換できること
 \end{itemize}
 などの理由から、3D写真撮影に有用な表現です。\\

 我々の手法への深度入力の品質は,カラーチャンネルと深度チャンネルの不連続性が適度に整列している限り,完全である必要はない.\\
実際には,デュアルカメラの携帯電話からの入力や学習ベースの手法からの推定深度マップ[30, 28]を用いて,我々の手法を成功裏に使用してきた.\\

\textbf{Method overview(手法の概要)}:



\begin{itemize}
  \item b
\end{itemize}


\begin{thebibliography}{99}
  \bibitem{dignity}
\end{thebibliography}



\end{document}
